\chapter[Cadeia de Suprimentos ]{Cadeia de Suprimentos}
\label{chap:cadeia}
	
	\section[Definição]{Definição}
	\label{sec:cadeia_definicao}

		A logística sempre representou um fundamental papel para as empresas, além de ser bastante centrada nas atividades tradicionais como: distribuição física, logística, armazenagem e estoques.
		Distribuição e logística
		
		A rede Bob’s contém em sua totalidade mais de 620 unidades espalhadas por todos os 26 estados brasileiros e estes importantes números a tornam a rede de fast-food com a maior cobertura geográfica no país. A rede de restaurantes Bob’s foi de fundamental importância na criação de pontos de venda móvel, sendo uma das pioneiras no mercado brasileiro. Implantaram inicialmente as mini lojas que eram feitas de aço com apenas 23 metros quadrados, que foram projetadas especificamente para ser colocada em qualquer lugar bastante movimentado como: praças, ruas e estacionamentos, e os chamados quiosques, atualmente com mais de 230 unidades de 15 metros quadrados, apropriadas para os corredores dos shoppings.
		
		Outro grande feito do Bob’s foi unir suas lanchonetes a postos de conveniência que hoje já totalizam 52 lojas, utilizando o conceito “store in store”, a rede descobriu um potencial de lucro ainda inexplorado pelas outras redes. Além disso, possui lojas nos principais aeroportos do Brasil que atendem 24 horas. O Bob’s possui além de tudo um moderno e sofisticado sistema de entrega de pedidos, que para ser realizado com eficiência necessita que o cliente se cadastre no site para a verificação de pedido e escolha da forma de pagamento, e também verificar se a sua residência encontra-se em uma das localidades atendidas dentro dos 12 estados que possuem o Bob’s Delivery. O Bob’s tem uma política justa de tempo máximo de entrega de 40 minutos para a entrega do pedido.
		
		Para abastecer de matéria-prima toda essa rede com lojas localizadas nos mais diversos lugares e continuar progredindo, a empresa, recentemente, resolveu melhorar seu sistema de distribuição. Com essa grande melhoria todas as regiões vão ter um sistema de abastecimento com qualidade equilibrada e de alto padrão.  A rede assinou um contrato com a TGB, uma grande empresa com especialidade no transporte de alimentos do tipo fast-food.Com esse novo sistema os custos de distribuição serão reduzidos em aproximadamente 20%. 

	\section[Cadeia de Suprimentos aplicada nas redes de fast-food do Bob's]{Cadeia de Suprimentos aplicada nas redes de fast-food do Bob's}
	\label{sec:cadeia_aplicadas}

		TO DO

