\chapter[Introdução]{Introdução}
\label{chap:introducao}
	
	O relatório será organizado em seções, ao qual pelo decorrer do curso, serão criadas. Isto é, o relatório será preenchido de acordo com cada atividade semanal. Isso ocorre pelo fato deste projeto ser gradativamente construído ao longo do semestre. Cabe ainda ressaltar que podem haver grandes mudanças em sua estrutura ou pequenas mudanças em tópicos específicos.

	\begin{enumerate}
		\item{\textbf{\nameref{chap:informacoesGerais}}: São apresentados os integrantes e o gestor do projeto, o tema escolhido para o projeto, além da ferramenta de gerenciamentos de atividades escolhida pela equipe;}
		\item{\textbf{\nameref{chap:processos}}: Caracteriza o projetos de processos para a rede Bob's;}
		\item{\textbf{\nameref{chap:produtos}}: Caracteriza os produtos e serviços oferecidos pela rede Bob's;}
		\item{\textbf{\nameref{chap:cadeia}}: Caracteriza a rede e cadeia de suprimentos do processos de produção da rede Bob's;}
		\item{\textbf{\nameref{chap:arranjo}}: Caracteriza arranjo físico e o fluxo do processos de produção da rede Bob's;}
		\item{\textbf{\nameref{chap:tecnologias}}: Caracteriza as tecnologias de materiais, informações e consumidores da rede Bob's;}
		\item{\textbf{\nameref{chap:organizacao}}: Caracteriza a organização do trabalho da rede Bob's}
		\item{\textbf{\nameref{chap:melhorias}}: Descreve as diretrizes propostas pela equipe para o processo de produção das redes Bob's;}
		\item{\textbf{\nameref{chap:consideracoes}}: Conclui o tema do relatório.}
	\end{enumerate}
