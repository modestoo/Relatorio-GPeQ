\chapter[Considerações Finais]{Considerações Finais}
\label{chap:consideracoes}
	
	Em 2010, a rede necessitava de um maior alcance entre os usuários da internet, porém, foi observado neste ano (2014) que a empresa está mais ativa nas redes sociais e possui um site com uma interface com padrões gráficos agradáveis, além de oferecer o serviço de fazer pedidos via internet. A rede \textbf{Bob's} passou por uma reestruturação recente e bem sucedida e com isso, foi capaz de se destacar-se no mercado brasileiro, portanto, ainda não é possivel projetar melhorias organizacionais significativas para ela, mas as melhorias que foram propostas nesse trabalho servem como um complemento no qual pode fazer com que a rede se destaque ainda mais, visto que a tecnologia NFC ainda é inutilizada entre as redes de \emph{fast food}.

	Essa reestruturação observada, expressa a importância de um prévio planejamento e análise aprofundados acerca do processo produtivo do produto/serviço da empresa, assim é possível evitar erros e problemas durante a execução das mudanças planejadas. Sabe-se ainda que a rede \textbf{Bob's} prevê uma nova reestruturação para os próximos anos, tangendo os aspectos organizacionais, de marketing e de produtos. Portanto, ela passa novamente por todo o processo de estudo, análise e implantação abordado parcialmente neste trabalho.

	Infelizmente não é possível analisar integralmente todo processo de produção da empresa por uma série de restrições da organização. No entanto, as pesquisas realizadas pela equipe conseguiram ser suficientes  para o entendimento do problema e estabelecimento de melhorias para o processo de produção.

	Futuras trabalho no qual podem ter um aprofundamento maior, poderão preencher algumas lacunas deixadas neste estudo, tais como as métricas para medição da qualidade de produção, gerando assim melhorias mais pontuais ao processo de produção, modelos do processo de produção atual e processo melhorado e as perspectivas a nível de marketing para os serviços oferecidos pela empresa no qual ainda carecem ser exploradas.


	