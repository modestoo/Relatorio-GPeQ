\chapter[Informações Gerais]{Informações Gerais}
\label{chap:informacoesGerais}
	
	Nesta seção serão abordados tópicos introdutórios aos desenvolvimento deste documento. Para isso, são apresentandos nos subtópicos a seguir as informações no qual expressam a ideia do projeto inicialmente.

	\section[Integrantes do Grupo]{\emph{Integrantes do Grupo}}
	\label{sec:informacoesGerais_integrantes}

		O grupo é composto por 6 colaboradores mais 1 gestor. A seguir são apresentados os nomes e suas respectivas matriculas.

		\label{subsubsec:informacoesGerais_integrantes_tables}
		\begin{table}[h]
			\centering 
			\begin{tabular}{r|c}

				Nome do Integrante & Matricula \\
				
				\hline

				Augusto Samuel Modesto & 12/0111314 \\
				Matheus Coelho & 12/0129345 \\
				Paulo Ananias de Sousa & 12/0131919 \\
				Renner Parente Magalhães & 13/0132101 \\
				Ruan Kevelin Neves & 12/0021978 \\
				Samantha de Oliveira Gil & 12/0135175 \\
				Willian Ricardo Coelho & 14/0166033 \\

			\end{tabular}
			\caption[Tabela de Integrantes do Grupo]{Tabela de Integrantes do Grupo.}
			\label{tab:informacoesGerais_integrantes_.tables}
		\end{table}

		Para a escolha do gestor, fora feito uma eleição para a escolha democrática do mesmo. Ao final da eleição, como representante do grupo, o \textbf{Augusto Samuel Modesto} foi elegido como o gestor do grupo.

	\section[Tema do Projeto]{\emph{Tema do Projeto}}
	\label{sec:informacoesGerais_tema}

		Para a escolha do tema, foram analisadas diversas temáticas. Entre as mais importantes estavam:

		\begin{enumerate}
			\item{\textbf{Produção de cerveja} – Produção de cerveja na empresa AMBEV;}
			\item{\textbf{Rede de Fast Food} – Rede de produção de hambúrgueres, milk shakes, entre outros;}
			\item{\textbf{Produção de Latinha de refrigerante} – Produção de latinhas na distribuidora empresa AMBEV.}
		\end{enumerate}

		Levando em consideração a facilidade de informações que podem ser obtidas nas redes de Fast Food, decidiu-se a escolha dela. Então, foram levantadas duas redes diferentes de fast food, o Giraffas e o Bob’s em diferentes localidades, neste caso, as franquias no aeroporto e no Taguatinga Shopping. A escolha de diferentes empresas servirá para geração de comparações entre o processo de produção de ambas as empresas e suas divergências com relação as suas localizações geográficas.

	\section[Ferramenta de Gestão]{\emph{Ferramenta de Gestão}}
	\label{sec:informacoesGerais_ferramenta}

		Para o controle das atividades, foram analisadas duas ferramentas para gerenciamentos das atividades produzidas pelo grupo. São elas:

		\begin{itemize}
			\item{\textbf{Trello};}
			\item{\textbf{ScrumMe}.}
		\end{itemize}

		Ambas as ferramentas possuim características muito próximas trabalhando com a ideia do \emph{Kanban} promovida pelas metodologias ágeis. Esse método permite uma fácil interação além de possuir uma curva de aprendizagem muito baixa, isso pode ser percebido ao longo da disciplina, uma vez que os integrantes conseguiram trabalho de forma eficiente na ferramenta escolhida.

		Avaliandos ambas as ferramentas, pode-se perceber que ambas possuiam um belo desempenho para gerenciamentos das atividades em geral. O \textbf{ScrumMe} apresentou uma excelente ferramenta de gráficos, no qual é possível verificar as atividades dos integrantes do grupo por meio de percentuais de trabalho, sendo possível validar quem trabalhou e quanto trabalho. No entanto, a ferramenta escolhida foi o \textbf{Trello} por ser muito mais simplista no seu funcionamento do que a outra. Esse critério foi adotado tendo em vista que a equipe de trabalho não possuia experiências na metodologia ágeis. Logo, uma ferramenta que facilita-se a aprendizagem em sua simplidade teria muito mais relevância do que alguns gráficos de desempenho.