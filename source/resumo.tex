\chapter[Resumo]{Resumo}
\label{chap:resumo}
	
	O processo de produção de um produto e/ou serviço é uma tarefa intensa que requer um alto nivel de conhecimento estratégico e organacional. Um dos papeis mais importante para o desenvolvimento desta atividade, o gerente de produção, acompanha de perto decisões de produção, tecnologias e negócios para que a empresa gerenciada tenha seus recursos bem explorados e retorno bem sucedido.

	Prejetar, analisar e sugir melhorias foram as etapas fundamentais para análise negocial da empresa \includegraphics{bobs2} estudada, localizada no Taguatinga Shopping. Para isso, apresentam-se breves resumos da livro Administração da Produção, de \cite{slack}, posteriomente as aplicações dos resumos ao contexto da empresa. Ao final, apresentam-se as melhorias propostas ao processo, além do questionário aplicado na entrevista a um dos gerentes de equipes.

	Vale ressaltar que partes das informações são adquiridas por meio de artigos pesquisados, devidamente referenciados de acordo com os padrões ABNT.

	\begin{flushleft}
		\textbf{Palavras-Chaves}: Processo de Produção, Gestão da Produção, Redes Bob's.
	\end{flushleft}
	